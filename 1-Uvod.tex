\newpage
\chapter*{Úvod} % normalni kapitola \chapter{kapitola}
\addcontentsline{toc}{chapter}{Úvod}

Za úkol na předmětu Projektové Praktikum bylo zadáno navrhnout a následně sestrojit aparaturu pro urychlování svazku elektronů a zařízení pro jejich detekci, což zahrnuje návrh a realizaci zapojení elektronového děla, zdroje vysokého napětí, doplňující soustavu fokusující svazek a v neposlední řadě samotný detektor intenzity elektronového svazku. 

Hned v úvodu jsme se tedy rozdělili na tři podskupiny, z nichž jedna se zabývala samotnou konstrukcí děla, jeho instalací do vakuové komory a zprovozněním urychlovací soustavy. Druhá skupina se zaměřila na fokusaci svazku elektronů, která spočívala v nalezení ideální konfigurace fokusovacích diod na základě simulací v programu SIMION. Třetí skupina se zabývala výrobou detektoru pro měření intenzity elektronového svazku, připojení vyčítacího zařízení a následné zpracování a analýza naměřených dat.

Na začátku projektu jsme se rozhodli, že se pokusíme vytvořit elektronový svazek o energii zhruba 80 keV, aby bylo možné použít monolitický pixelový detektor, který jsme si byli schopni sami sestrojit. První pokusy vytvořit elektronový svazek se žhaveným wolframovým vláknem jako zdrojem elektronů však nebyly úspěšné a později jsme přešli k průmyslovému elektronovému dělu. Svazek jsme urychlovali měděnými elektrodami. Z důvodu pozorovaných elektrických výbojů jsme nebyli na elektrodách schopni dosáhnout zamýšleného potenciálového rozdílu 80 kV, nýbrž jen 18 kV. K detekci signálu jsme proto použili destičku ze slitiny mědi a zlata, která konvertovala elektrony na námi detekovatelné fotony, a analyzovali výsledek měření.

První kapitola poskytuje teoretický úvod o elektronech a jejich chování v prostředí se sníženým tlakem. Je zde také popsán princip katodového záření a možnosti vedení elektrického proudu ve vakuu. 

Druhá kapitola pojednává o základních principech vzniku elektrických výbojů a jejich dělením. Druhá část této kapitoly se věnuje problematice vzniku elektrických výbojů v našem experimentu a popisuje způsoby, kterými jsme se snažili výbojům zamezit a předejít.

Třetí kapitola se zabývá principem fungování elektronového děla a emisí elektronů. Dále je zde popsáno sestavení vlastního elektronového děla a jeho technické parametry.

Ve čtvrté kapitole je představen zdroj elektronů ES40-PS, kterým bylo v pozdější fázi projektu nahrazeno naše původní elektronové dělo. V kapitole jsou rozepsány především technické detaily tohoto zdroje a jeho správné ovládání.

Pátá a šestá kapitola jsou věnovány fokusaci elektronového svazku. První ze zmiňovaných kapitol se zabývá především teorií samotné fokusace. Druhá zmíněná kapitola pak představuje simulační program SIMION a jeho konkrétní využití v našem experimentu.

Předposlední kapitola popisuje principy detekce elektronů a představuje detektor použitý v našem experimentu, jeho výrobu a technické parametry.

V poslední kapitole je uvedena metodika měřením, naměřená data a výsledky analýzy dat.

