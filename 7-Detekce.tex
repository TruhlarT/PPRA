\newcommand{\fig}[3]{ 
	\begin{figure}[!ht]
		\centering
		\includegraphics[width=#1\linewidth]{Figure/07/#2.png}
		\caption{#3}
		\label{#2}
	\end{figure}
}

\interfootnotelinepenalty=10000
\renewcommand{\figurename}{Obr.}
\newcommand{\figref}[1]{\figurename \hspace{1 pt} \ref{#1}}

\newpage \clearpage

\chapter{Detekce svazku}

\section{Detekční technologie}
Po zvážení možností způsobu detekce elektronového svazku bylo jako optimální možnost zvoleno použití křemíkového pixelového detektoru X-CHIP03 vyvinutého v rámci FJFI ČVUT v Praze skupinou CAPADS. Ten by měl, jakožto detektor ionizujícího záření, dobře posloužit účelu, tedy zaznamenávání průletu elektronů. Další výhodou jeho použití byla možnost konzultace s týmem, který jej vytvořil. 

Vzhledem k monolitické technologii, v níž je detekční čip navržen, existuje spodní hranice pro energii detekovaných částic. Ta byla pro elektrony ze simulací stanovena na 60 keV - hodnotu, jíž bylo teoreticky elektronovým dělem dosáhnout. Vzhledem ke komplikacím, které se objevily během testování (dříve zmíněné vakuové výboje, ...) se nakonec bohužel podařilo dosáhnout energií maximálně 15 keV, které neprojdou vrstvou elektroniky do citlivé oblasti.

K vyřešení tohoto problému bylo navrženo využití konverzní vrstvy slitiny mědi a zlata. Tato tenká folie byla umístěna do cesty svazku před detektorem a při ozařování elektrony docházelo k emisi fotonů o podobné energii, které už nebyl problém detekovat. 

\section{Principy polovodičových detektorů}
\subsection{Pásová struktura}
Krystalické pevné látky se klasifikují jako vodiče, polovodiče a izolanty, přičemž kriteriem pro toto rozdělení je tzv. šířka zakázaného pásu. Ta vyplývá z pásové struktury (\figref{pasova_struktura}), tedy modelu, který popisuje energetické rozdělení elektronů v krystalu. 

\fig{1}{pasova_struktura}{Grafické znázornění pásové struktury a) izolantu, b) polovodiče a c) vodiče. \cite{lutz1999semiconductor}}

Energetické spektrum samostatného elektronu je diskrétní, přičemž jednotlivé hladiny jsou určeny kvantovými čísly. Přidáváním dalších elektronů postupně dochází k rozštěpení hladin a při přechodu k makroskopickému popisu krystalu, je již třeba použít statistický model. Tím je pásová struktura, ve které se při velkém počtu elektronů původně diskrétní energetické hladiny shlukují do spojitých pásů - valenčního a vodivostního.

\textit{Valenční pás} obsahuje energie, při kterých jsou elektrony vázané v atomovém obalu a nemohou se volně pohybovat látkou. 

Přijme-li elektron dostatečnou energii, může přejít do \textit{vodivostního pásu}, ve kterém je v rámci krystalické mřížky delokalizovaný a může se podílet na vedení elektrického proudu.

Je-li krystalická látka vodivá, oba tyto pásy se překrývají a elektrony se tak mohou podílet na vedení proudu přirozeně. 

V případě izolantů jsou naopak pásy oddělené širokou oblastí energií, kterých elektrony nemohou nabývat - tzv. zakázaným pásem. Jeho šířka v prvním přiblížení\footnote{Roli zde hrají ještě další faktory, které oproti samotné ionizační energii šířku pásu zvětšují.} odpovídá ionizační energii atomů. 

Polovodiče mají sice ve své struktuře zakázaný pás také, jeho šířka je však na rozdíl od izolantů menší (řádově v jednotkách eV). To umožňuje korigovat jejich vlastnosti tak, aby podle potřeby vedly nebo nevedly elektrický proud. Když dojde k ionizaci atomu, vytvoří se v polovodiči dva efektivní nosiče náboje. Elektrony, které jsou na \figref{pasova_struktura} zobrazeny tečkami ve vodivostním pásu a díry, tedy absence elektronů v páse valenčním, zobrazené prázdnými kroužky.

Pohyb elektronů a děr společně vytváří celkový proud, přičemž elektrony se v materiálu pohybují rychleji\footnote{V křemíku je mobilita elektronů oproti děrám $\sim 3 \times$ větší.}.

\subsection{Vyprázdněná oblast}
Vlastnosti polovodičů mohou být zlepšeny tzv. dopováním. Při tomto procesu dochází k nahrazení malého množství atomů původní látky atomy s jiným (ale velmi blízkým) protonovým číslem. Tím vznikne přebytek nebo naopak nedostatek elektronů v krystalové mřížce\footnote{Polovodiče typu N mají přebytek elektronů, typu P přebytek děr.} a jsou tak apriori přítomny volné nosiče náboje.

Připojíme-li k sobe opačně dopované polovodiče, vznikne dioda a na jejím rozhraní tzv. vyprázdněná oblast, viz \figref{depleted}. Jde o místo, ze kterého jsou volné nosiče náboje vypuzeny vlivem přirozeně vzniklého elektrického pole. Pokud je navíc na diodu přivedeno napětí v závěrném směru, šířka vyprázdněné oblasti se zvětší.

\fig{.9}{depleted}{Přechod mezi P (nalevo) a N (napravo) polovodiči. Znaménka v kroužku značí volné nosiče náboje, které jsou v přítomnosti pole odpuzeny, nezakroužkovaná znaménka pak ionty ve vrcholech krystalické mřížky. Graf pod obrázkem zobrazuje průběh odpovídajícího potenciálu. \cite{spieler2005semiconductor_detector_systems}}

%Základním prvkem polovodičového detektoru je tzv. vyprázdněná oblast. Jde o místo s minimální koncentrací volných nosičů náboje, které vznikne v diodě při jejím zapojení v závěrném směru, viz \figref{depleted}. Na diodu je dále přivedeno tzv. biasovací napětí, které dále rozšiřuje vyprázdněnou oblast a navíc urychluje volné nosiče náboje, které by v ní vznikly.

\subsection{Vznik signálu}
%Pakliže vyprázdněnou oblastí proletí ionizující částice, předávají energii vázaným elektronům, které přejdou do vodivostního pásu. Při tomto přechodu vzniknou dva volné nosiče náboje: volný elektron a tzv. díra, tedy vakance ve valenčním páse, která se efektivně chová jako volný nosič náboje opačného znaménka. 

Pokud proletí elektricky nabitá částice vyprázdněnou oblastí předávají energii vázaným elektronům. Tak může dojít k předání energie dostatečné k přechodu vázaného elektronu do vodivostního pásu, tedy ionizaci atomu. Tím vznikne elektron-děrový pár.

Pokud má částice dostatečnou energii, mohou tyto páry vznikat podél celé její trajektorie, jak je vidět na \figref{particle-detection}. V elektrickém poli\footnote{Pole vzniká jako přirozený důsledek závěrného zapojení diody v kombinaci s biasovacím napětím.} driftují směrem ke sběrným elektrodám, které mohou bít umístěny na přední a zadní straně senzitivní oblasti.

Driftem elektronů a děr vzniká v materiálu proud, který je následně registrován jako signál. Jeho časový vývoj závisí například na typu nosiče\footnote{Elektrony se v křemíku pohybují $\sim 3 \times$ rychleji, než díry.} nebo na jeho aktuální poloze, pak záleží na tvaru elektrického pole v souladu se Shockley-Ramovým teorémem \cite{ramo1939}, \cite{shockley1938}. Příklad časové závislosti signálu je na \figref{pd_weightfield_mu}.

\fig{1}{particle-detection}{Znázornění elektron-děrových párů vzniklých ve vyprázdněné oblasti detektoru při průletu různých druhů ionizujících částic.}

\fig{1}{pd_weightfield_mu}{Časový vývoj proudového signálu generovaného mionem o energii 1 GeV v pixelu křemíkového. Model z programu Weightfield2 \cite{2015weightfield}. Oranžová křivka značí proud indukovaný pohybem děr,modrá pohybem elektronů a černá celkový.}

\clearpage

\section{Pixelové detektory}
Polovodičové detektory se podle strukrtury uspořádání senzitivních oblastí dělí na stripové a pixelové. 

U prvních zmíněných je senzitivní oblast rozdělena na proužky (stripy), což je technologicky snazší varianta, detektor ale tak poskytuje informaci o zásahu pouze v jednom rozměru. 

Pixelové detektory naproti tomu mají senzitivní oblast tvořenou maticí pixelů, díky čemuž je jejich prostorové rozlišení apriori dvourozměrné. Pro účely skenování průřezu svazku je tedy pixelový detektor ideální volbou. Navíc se nabídla možnost využití pixelových detektorů vyvíjených na katedře, díky čemuž bylo možné v průběhu práce konzultovat jejich aplikaci s designéry.

V průběhu práce na experimentu se bohužel ukázalo, že výhody 2-D rozlišení není možné dostatečně využít, protože se nepodařilo svazek fokusovat do dostatečně malého průřezu a detektor tak vždy snímal pouze jeho část. Volbu pixelového detektoru však přesto považujeme vzhledem k detektoru za nejlepší možnou variantu.

%Jedním z typů polovodičových detektorů jsou pixelové, tedy matice nezávislých detekčních jednotek. Oproti druhé nejvýznamnější rodině - stripovým - nabízí apriori 2D rozlišení což je pro účel příčného skenování  elektronového svazku ideální. Tato potenciální výhoda bohužel nakonec zůstala nevyužita, protože se nepodařilo svazek dostatečně fokusovat a v místě detekce tak pokrýval větší plochu, než samotný detektor.\footnote{Je také možné, že k defokusaci svazku došlo využitím konverzní vrstvy.}

\section{X-CHIP03}
Pro skenování elektronového svazku byl v rámci tohoto projektu zvolen X-CHIP03 (\figref{xchip03_mp}). Jde o monolitický\footnote{V případě monolitických detektorů je na rozdíl od hybridních vyčítací část elektroniky umístěna ve stejném jednolitém objemu křemíku jako senzitivní vyprázdněná oblast.} pixelový detektor tvořený maticí 64x64 pixelů o rozměrech $4.2\times 5 \mathrm{mm}^2$. 

\fig{1}{xchip03_mp}{Model designu (vlevo) a fotografie (vpravo) detekčního čipu X-CHIP03, \cite{xchip03}.}

Detekční set-up, viz \figref{setup} se skládá z čipu umístěného na FURRy, PCB zprostředkující transfer fyzického signálu do USB portu, který je možné přímo připojit k počítači. Ke zpracování dat pak slouží software ASPIRE, přičemž obě zmíněné složky byly vyvinuty v rámci CAPADS. Celý set-up je umístěn na stojanu, který byl pro tento účel navržen a vytvořen ve 3-D tiskárně. Přechod mezi vakuem a notebookem vyčítajícím data byl zajištěn přes přechodku vakuové komory, na kterou byl z obou stran zajištěn rozpojený USB kabel.

\fig{1}{setup}{Experimentální detekční set-up. V pravé části se nachází detekční čip X-CHIP03, překrytý zlatavou konverzní folií. Větší PCB je FURRy, oranžové komponenty slouží jako stojánek, dále upevněný ve vakuové komoře. Z levé části rovněž vystupuje port USB.}

\label{chapter2}

\clearpage \pagebreak \newpage