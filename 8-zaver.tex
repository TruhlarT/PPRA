\newpage
\chapter*{Závěr} % normalni kapitola \chapter{kapitola}
\addcontentsline{toc}{chapter}{Závěr}

Hlavním cílem této práce byla konstrukce zdroje elektronů, urychlovací a fokusovací aparatury spolu s detekčním zařízením a následné proměření vlastností celého zařízení a samotného elektronového svazku. 

V této práci byly nejprve představeny základy teorie elektronů a jejich chování ve vakuu. Dále byly popsány elektrické výboje a způsoby jejich eliminace. Ve třetí kapitole byly představeny metody získávání elektronů spolu s námi vytvořenými elektronovými zdroji a urychlovací aparaturou. V následující kapitole byl dále popsán zdroj elektronů ES40-PS a jeho technické parametry spolu s návodem na jeho ovládání. V páté kapitole byly popsány teoretické principy fokusace elektronového svazku. V následující kapitole jsme se seznámili se simulačním programem SIMION a jeho konkrétním využitím v našem experimentu spolu s výsledky měření fokusace elektronového svazku, které demonstrují funkčnost celé aparatury. V předposlední kapitole byly představeny principy detekce elektronů a námi použitý detektor spolu s jeho výrobou a technickými parametry. Nakonec byla představena metodika měření, naměřená data a výsledky analýzy dat, kde jsme konverzí elektronů na fotony pomocí slitiny mědi a zlata změřili spodní limit energie elektronů jako 8 keV.

